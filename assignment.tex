\documentclass[10pt,english]{article}
%\documentclass[10pt,italian]{article} % TO WRITE IN ITALIAN, UNCOMMENT THIS LINE AND COMMENT THE ABOVE ONE
\usepackage[utf8]{inputenc} % opzione per caratteri ISO-8859-1, CONSENTE L'USO DELLE ACCENTATE
\usepackage{hyperref}

% MARGINI LARGHI
\textwidth 6.3 in % Width of text line.
    \textheight 9.2 in
    \oddsidemargin 0 in      %   Left margin on odd-numbered pages.
    \evensidemargin 0 in      %   Left margin on even-numbered pages.
    \topmargin 0.2 in
    \headheight 0 in       %   Width of marginal notes.
    \headsep 0 in
    \topskip 0 in
    
\title{\vspace{-70pt}Assignment 03 -- ``Prolog''\footnote{Produce a documentation according to this template. Try to keep this document in one page, overall work no more than 6-7 hours, do the job yourself, deliver before the exam, consider the possibility of doing it this very week. Erase this and all other footnotes at the end before submitting.}}
\author{Riccardo Soro, matr: 851004, email: {\url{riccardo.soro@studio.unibo.it}}\\ repo: {\url{https://github.com/riccardosoro/pps2017-asg03-prolog.git}}\footnote{You have to implement some nice code that involves Prolog (be excellent and rely on logic programming, also exercise a multi-paradigm approach if you can): it can be some solution to problems given in lab or some variation of it, it can be some new Prolog code, a mini-JVM-application demoing Java/Scala/Prolog, it can be a better Scala wrapper for tuProlog, or it can be everything else you want and like (just stay within 6-7 hours of work). Please name your project exactly \texttt{pps2017-asg03-prolog}, and perform meaningful commits. Be sure the repo is visible and do not modify it after submission.}}
\date{04/06/2018}


\begin{document}

\maketitle
\vspace{-30pt}

\subsection*{Description of the code you provide\footnote{As example we here indicated some possibilities.. depending on time, you could do one or more things.}}

\begin{itemize}
 \item Full solution to labs 10 and 11, no tic tac toe
 \item \ldots
\end{itemize}

\subsection*{Techniques used\footnote{Please list the LP-related features you believe you have correctly exercised, and hence, understood.}}

\begin{itemize}
 \item Prolog search-space abilities
 \item Prolog use of unification and variables
 \item Java-tuProlog integration
 \item Compared monads and LP
 \item \ldots
\end{itemize}

\subsection*{Self-evaluation\footnote{Add a max 10 lines evaluation of this experience, reporting what you think about Prolog and LP, about what you have learned of it, and about this specific assignment: what went good, what bad, and so on.}}

\ldots 
 
\end{document}
    
    
